\cleardoublepage
\phantomsection
\addcontentsline{toc}{chapter}{Introduction}
\chapter*{Introduction}
\chapterquote{``If a question can be put at all, \\ then it \emph{can} also be answered.''
}{Ludwig Wittgenstein}
%
%
\par{In the context of new developments in molecular physics and chemistry studying
light emission by molecules is one of the most interesting jobs,
nowadays. There are still few unanswered questions concerning theirs
physics. We will try to focus ourselves on two of them. Firstly, the mechanism
of the photoelectron transport along saturated carbon bonds and secondly, the
rate of fluorescence of a conventional molecular dye. These are two of the
features which can not be satisfactory explained using first principles
methods. In our group former research in theoretical physics led the group in
the position of developing two new approaches for tight binding -
tight-binding approximation extended  to a self consistent theory with
polarisable atoms \citep{Finnis98} and a new formulation of time dependent tight-binding
\citep{Todorov01}. We intend to merge these two approaches in a new and original view. TB+U
is a new many-body formulation of tight-binding theory developed in our
group.}
\par{Time dependent Schr{\"o}dinger equation will allow us to study these
problems. The interest in these kind of problems is sustained by the
importance of push-pull chromophores for the biological processes \citep{deSilva01} and
making logic devices using their fluorescent \emph{on-off} features \citep{deSilva01}. We will
test our approach on a very well known push-pull molecule
para-nitro aniline. Being a relative small molecule, only 16 atoms, and well
investigated by experimental chemistry should give us the insight for testing
and improving our theory.}
\par{Another very important step is the implementation of this theory in a computer
code. The complexity of the systems intended to study does the parallel
programming approach a natural one. Time propagation scheme could be crucial,
our aim is to simulate for a long time, so a careful investigation of the
proper algorithm is imposed. Once tested and implemented we will try to apply
our method to more interesting molecules in PET sensors family \citep{deSilva01b}.}
\par{As a general outline}
\begin{itemize}
\item become familiar with general tight binding schemes
\item study and become familiar with different parallel programming
  environments -- MPI, OpenMP -- and techniques
\item DFT study of $pNA$ and obtaining the tight binding parameters
\item implement in a code the equation of motion of \citep{Todorov01}
\item testing and choosing a time propagation scheme
\item proposing a time dependent tight binding model suitable for PET
  molecules
\item employing the proposed model to study PET sensors of practical importance
\end{itemize}
\par{ This is a short introduction in what we expect to accomplish in this  project. The report presented next wants to show the progress in our  study.}
\par{The report it is organised as follows. First part is a short (maybe too short)
introduction in tight binding theory and a general presentation of a
Goodwin-Skinner-Pettifor tight binding based model chosen for our
study. Second part will deal with Slater-Koster parametrisation. It will
present a general approach of automatic generating of what is referred as
Slater-Koster tables and theirs first and second derivatives. Third part will
present some technical details about fitting the parameters in our GSP
model. The fitting could be reduced very easy to a minimisation of an
objective function. Three methods are employed to solve this problem Simplex,
Simulated Annealing (SA) and a hybrid method Simplex-Simulated Annealing
(SSA). Tests results for these algorithms are presented and $N-H$ and $N-N$
parameters are proposed. Some issues about the transferability of this
parameters are risen.}