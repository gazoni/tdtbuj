\section[Complex Spherical ...]{Complex Spherical Harmonics}
\subsection{Definition}
 \par{A spherical harmonic $Y_{lm}(\vartheta,\varphi)$ is a single-valued function,
continuous, bounded complex function of two real arguments $\vartheta,\varphi$ with
$0\le \vartheta \le \pi$ and $0\le \varphi <2\pi$. It is characterised by the
parameters $l,m$, which take values $l=0,1,2,...$ and $|m| \le l$. Therefore, for
a given $l$ there are $2l+1$ functions corresponding to different $m$'s. All
derivatives of $Y_{lm}(\vartheta,\varphi)$ are single-valued, continuous and finite
functions.}
\subsection{Differential Equations}
\par{They are important in quantum physics. In fact they are eigenfunctions of
the operator of orbital angular quantum momentum and describe the angular
distribution of particles which move in a spherically-symmetric field with
orbital angular momentum $l$ and projection $m$. If ${\bf L}^2$ is the square
of orbital angular momentum and ${\bf L}_z$ the projection of orbital angular
momentum on the quantization axis we can write}
 
\begin{equation}
 \begin{split}
 {\bf L}^2Y_{lm}(\vartheta,\varphi)&=l(l+1)Y_{lm}(\vartheta,\varphi)\\
 {\bf L}_zY_{lm}(\vartheta,\varphi)&=mY_{lm}(\vartheta,\varphi)
 \end{split}
 \end{equation}
or in an expanded form
\begin{equation}
\label{sphdiff}
\begin{split}
\left[\frac{1}{\sin \vartheta}\p{}{\vartheta}\left(\sin \vartheta
    \p{}{\vartheta}\right)+\frac{1}{\sin^2
    \vartheta}\pp{}{\varphi}+l(l+1)\right]Y_{lm}(\vartheta,\varphi)&=0\\
\left[i\p{}{\varphi}+m\right]Y_{lm}(\vartheta,\varphi)&=0
\end{split}
\end{equation}
\par{Equations \gref{sphdiff} are invariant under the following transformations}
\begin{itemize}
\item $l\rightarrow \overline{l}=-l-1$
\item $\vartheta \rightarrow -\vartheta$ or $\vartheta \rightarrow
  \pi-\vartheta$
\item $m\rightarrow -m$
\item $\varphi \rightarrow -\varphi$
\end{itemize}
\subsection{Boundary Conditions}
\par{The first equation of \gref{sphdiff} is of second order. For fixed $l$ and $m$
it has two linear independent solutions. However, only one of them is regular,
satisfies the condition $|Y_{lm}(\vartheta,\varphi)|^2<\infty$ while the
other is singular at $\vartheta=0,\pi$. The regular solution satisfies the
following boundary conditions}
\begin{equation}
\label{boundary}
\begin{split}
Y_{lm}(\vartheta,\varphi \pm 2n\pi )&=Y_{lm}(\vartheta,\varphi)\\
\left.\p{}{\varphi}Y_{lm}(\vartheta,\varphi)\right|_{\vartheta=0}&=\left.\p{}{\varphi}Y_{lm}(\vartheta,\varphi)\right|_{\vartheta=\pi}=0
\end{split}
\end{equation}
\par{We shall consider only the spherical harmonics $Y_{lm}(\vartheta,\varphi)$
with $l$ and $m$ integers because the boundary conditions \gref{boundary} are
fulfilled only for such values of the parameters.}
\subsection[Normalization, Completeness ...]{Normalization, Completeness and other relations}
\par{The differential equations \gref{sphdiff} and the boundary conditions
  \gref{boundary} are homogeneous. Hence, they determine the spherical
  harmonics up to an arbitrary complex factor. The absolute value of this
  factor could be fixed by ortho-normalization relation}
\begin{equation}
\int_{0}^{2\pi}\td \varphi\int_{0}^{\pi}\td \vartheta \sin \vartheta Y_{l_1m_1}^{*}(\vartheta,\varphi)Y_{l_2m_2}(\vartheta,\varphi)=\delta_{l_1l_2}\delta_{m_1m_2}
\end{equation}
\par{The completeness relation for the spherical harmonics id given by}
\begin{equation}
\label{complet}
\sum_{l=0}^{\infty}\sum_{m=-l}^{m=l}Y_{lm}^{*}(\vartheta_1,\varphi_1)Y_{lm}(\vartheta_2,\varphi_2)=\delta(\varphi_1-\varphi_2)\delta(\cos
\vartheta_1-\cos \vartheta_2)
\end{equation}
\par{Sometimes instead of $Y_{lm}(\vartheta,\varphi)$ it is more convenient to
  use the function $C_{lm}(\vartheta,\varphi)$} which differs from
$Y_{lm}(\vartheta,\varphi)$ by the normalization factor
\begin{equation}
C_{lm}(\vartheta,\varphi)=\sqrt{\frac{4\pi}{2l+1}}Y_{lm}(\vartheta,\varphi) 
\end{equation}
\par{The function $C_{lm}(\vartheta,\varphi)$ satisfies the following relation}
\begin{equation}
\sum_{m}C_{lm}(\vartheta,\varphi)C_{lm}^{*}(\vartheta,\varphi)=1
\end{equation}
\par{The ortho-normalization relation is}
\begin{equation}
\int_{0}^{2\pi}\td \varphi\int_{0}^{\pi}\td \vartheta \sin \vartheta C_{l_1m_1}^{*}(\vartheta,\varphi)C_{l_2m_2}(\vartheta,\varphi)=\frac{2l+1}{4\pi}\delta_{l_1l_2}\delta_{m_1m_2}
\end{equation}
\subsection{Choice of Phase}
\par{An overall phase factor may be fixed by specifying the phase of one of the
harmonics $Y_{lm}(\vartheta,\varphi)$ for some given values of the arguments,
for example}
\begin{equation}
\label{y00}
Y_{l0}(0,0)=\sqrt{\frac{2l+1}{4\pi}} 
\end{equation}
\par{In this case the following relations are valid}
\begin{equation}
\label{ycomplex}
Y_{lm}^{*}(\vartheta,\varphi)=Y_{lm}(\vartheta,-\varphi)=(-1)^{m}Y_{l-m}(\vartheta,\varphi)
\end{equation}
\par{This choice of phase is known as Condon-Shortley phase convention and is
  very common in physics.}
\par{Equations \gref{sphdiff} and relations \gref{boundary}, \gref{y00},
  \gref{ycomplex} completely define $Y_{lm}(\vartheta,\varphi)$. Since $l$ and $m$ are integers,
  $Y_{lm}(\vartheta,\varphi)$ is single-valued.}
\subsection[Solutions of Some ...]{Solutions of Some Differential Equations in Terms of $Y_{lm}(\vartheta,\varphi)$}
\par{(a) Solution of Laplace equation}
\begin{equation}
\nabla^2f(r,\vartheta,\varphi)=0
\end{equation}
in polar coordinates is given by
\begin{equation}
R_{lm}(r,\vartheta,\varphi)=r^lY_{lm}(\vartheta,\varphi)\qquad I_{lm}(r,\vartheta,\varphi)=r^{(-l-1)}Y_{lm}(\vartheta,\varphi) 
\end{equation}
where $I_{lm}(r,\vartheta,\varphi)$ is irregular at $r=0$ and
$R_{lm}(r,\vartheta,\varphi)$ is regular. These functions are called solid
harmonics. In Cartesian coordinate representation $R_{lm}$ is a homogeneous
polynomial of degree $l$
\begin{equation}
r^lY_{lm}(\vartheta,\varphi)=\sqrt{\frac{2l+1}{4\pi}(l+m)!(l-m)!}\sum_{p,q,t}\frac{1}{p!q!t!}\left(-\frac{x+iy}{2}\right)^p\left(\frac{x-iy}{2}\right)^qz^t
\end{equation}
\par{Here $p,q,t$ are positive integers which satisfy $p+q+t=l$, $p-q=m$.}
\par{(b) Solution of the Helmholtz wave equation}
\begin{equation}
\left[\nabla^2+k^2\right]f(r,\vartheta,\varphi)=0
\end{equation}
\par{In polar coordinates may be expressed in terms of the function
$z_l(kr)Y_{lm}(\vartheta,\varphi)$ where
$z_l(kr)=\sqrt{\frac{\pi}{2kr}}Z_{l+1/2}(kr)$, $Z_{l+1/2}(kr)$ being any of
Bessel functions.}
\begin{equation}
L_{lm}^{r}(r,\vartheta,\varphi)=i^lj_l(kr)Y_{lm}(\vartheta,\varphi)\qquad L_{lm}^{i}(r,\vartheta,\varphi)=i^ln_l(kr)Y_{lm}(\vartheta,\varphi) 
\end{equation}
\par{$L_{lm}^{r}$ is regular at $r=0$, whereas $L_{lm}^{i}$ is irregular. These
functions are called standing spherical waves.}
\begin{equation}
B_{lm}^{(1)}(r,\vartheta,\varphi)=i^lh_l^{(1)}(kr)Y_{lm}(\vartheta,\varphi)\qquad B_{lm}^{(2)}(r,\vartheta,\varphi)=i^lh_l^{(2)}(kr)Y_{lm}(\vartheta,\varphi) 
\end{equation}
\par {$B_{lm}^{(1)}$ corresponds to a spherical wave which converges to origin, while
$B_{lm}^{(2)}$ corresponds to an outgoing spherical wave. These functions are
called running spherical waves. In the limit $k \rightarrow 0$ we regain
Laplace equation and solid harmonics.}
\subsection[Explicit form ...]{Explicit form of spherical harmonics}
\begin{equation}
Y_{lm}(\vartheta,\varphi)=e^{im\varphi}\sqrt{\frac{2l+1}{4\pi}\frac{(l-m)!}{(l+m)!}}P_l^m(\cos \vartheta)
\end{equation}
where $P_l^m(\cos \vartheta)$ associated Legendre polynomials.
\par{For special values of arguments and parameters we have}
\begin{equation}
Y_{lm}(0,\varphi)=\delta_{m0}\sqrt{\frac{2l+1}{4\pi}}
\end{equation}
\begin{equation}
Y_{lm}(\pi,\varphi)=\delta_{m0}(-1)^l\sqrt{\frac{2l+1}{4\pi}}
\end{equation}
\begin{equation}
Y_{lm}(\pm n\pi,\varphi)=\delta_{m0}(-1)^{nl}\sqrt{\frac{2l+1}{4\pi}}
\end{equation}
\begin{equation}
Y_{lm}(\frac{\pi}{2},\varphi)=\left\{\begin{array}{lc}
(-1)^{\frac{l+m}{2}}e^{im\varphi}\sqrt{\frac{2l+1}{4\pi}\cdot
  \frac{(l+m-1)!!}{(l+m)!!}\cdot \frac{(l-m-1)!!}{(l-m)!!}} & \text{if
}l+m\text{ is even}\\
0 & \text{if }l+m\text{ is odd}
\end{array}
\right.
\end{equation}
\subsection[Symmetry Properties ...]{Symmetry Properties and Other Relations}
\par{The symmetry relations given bellow couple spherical harmonics
  $Y_{lm}(\vartheta,\varphi)$ with different values of $\vartheta,\varphi$ and
  $l,m$.}
\begin{equation}
Y_{lm}^{*}(\vartheta,\varphi)=Y_{lm}(\vartheta,-\varphi)=(-1)^mY_{l-m}^{*}(\vartheta,\varphi)
\end{equation}
\begin{equation}
Y_{l-m}(\vartheta,\varphi)=(-1)^mY_{lm}(\vartheta,-\varphi)=(-1)^mY_{lm}^{*}(\vartheta,\varphi)=(-1)^me^{-i2m\varphi}Y_{lm}(\vartheta,\varphi)
\end{equation}
\begin{equation}
Y_{-l-1,m}(\vartheta,\varphi)=(-1)^mY_{lm}(\vartheta,\varphi)
\end{equation}
\begin{equation}
Y_{lm}(\pi-\vartheta,\varphi)=(-1)^{l+m}Y_{lm}(\vartheta,\varphi)
\end{equation}
\begin{equation}
Y_{lm}(\vartheta,\pi+\varphi)=(-1)^mY_{lm}(\vartheta,\varphi)
\end{equation}
\begin{equation}
Y_{lm}(\pi-\vartheta,\pi+\varphi)=(-1)^lY_{lm}(\vartheta,\varphi)
\end{equation}
\begin{equation}
Y_{lm}(-\vartheta,\varphi)=(-1)^mY_{lm}(\vartheta,\varphi)
\end{equation}
\begin{equation}
Y_{lm}(\vartheta,-\varphi)=(-1)^mY_{l-m}(\vartheta,\varphi)
\end{equation}
\begin{equation}
Y_{lm}(-\vartheta,-\varphi)=Y_{l-m}(\vartheta,\varphi)
\end{equation}
\begin{equation}
Y_{lm}(\vartheta \pm n\pi,\varphi)=\left\{ \begin{array}{lc}
(-1)^lY_{lm}(\vartheta,\varphi) & \text{if }n \text{ is odd}\\
Y_{lm}(\vartheta,\varphi) & \text{if }n \text{ is even}\\
\end{array}
\right.
\end{equation}
\begin{equation}
Y_{lm}(\vartheta,\varphi \pm n\pi)=\left\{ \begin{array}{lc}
(-1)^mY_{lm}(\vartheta,\varphi) & \text{if }n \text{ is odd}\\
Y_{lm}(\vartheta,\varphi) & \text{if }n \text{ is even}\\
\end{array}
\right.
\end{equation}
\par{Other useful relations could be}
\begin{equation}
\sum_{m} Y_{lm}(\vartheta,\varphi)Y_{lm}{*}(\vartheta,\varphi)=\frac{2l+1}{4\pi}
\end{equation}
\par{Proof is easy. We start with}
\begin{equation}
\sum_{m} Y_{lm}(\vartheta,\varphi)Y_{lm}{*}(\vartheta,\varphi)=I
\end{equation}
and then integrating over the angles and using the completeness relation we get
\begin{equation}
\begin{split}
\sum_{m}&=4I\pi\\
I&=\frac{2l+1}{4\pi}
\end{split}
\end{equation} 
\begin{equation}
\sum_{m} mY_{lm}(\vartheta,\varphi)Y_{lm}{*}(\vartheta,\varphi)=0
\end{equation}
\par{Following the same line as for the previous relation we get}
\begin{equation}
\begin{split}
\sum_{m=-l}^{l} m&=I\\
I&=0
\end{split}
\end{equation}
\par{A comprehensive description of complex spherical harmonics and their
  properties can be found in \citep{Varshalovich88}} 
%
\subsection[Expansion in Series ...]{Expansion in Series of the Spherical Harmonics}
%
\par{An arbitrary function $f(\vartheta,\varphi)$ which is defined in the interval $0\le \vartheta \le
\pi$ and $0\le \varphi <2\pi$ and satisfies the condition}
\begin{equation}
\int_{0}^{2\pi}\td \varphi \int_{0}^{\pi} \td \vartheta \sin \vartheta
|f(\vartheta,\varphi)|^2< \infty
\end{equation} 
can be expanded into a series of the spherical harmonics as
\begin{equation}
\label{multipole}
f(\vartheta,\varphi)=\sum_{l=0}^{\infty}\sum_{m=-l}^{m=l}a_{lm}Y_{lm}(\vartheta,\varphi)
\end{equation}
\par{The expansion coefficients $a_{lm}$ are given by the relation}
\begin{equation}
a_{lm}=\int_{0}^{2\pi}\td \varphi \int_{0}^{\pi} \td \vartheta \sin \vartheta Y_{lm}^{*}(\vartheta,\varphi)f(\vartheta,\varphi)
\end{equation}
\par{This relation may be treated as an integral transformation of
$f(\vartheta,\varphi)$ from the continuous variables $\vartheta,\varphi$ to
the discrete variables $l,m$.}
\par{The expansion coefficients $a_{lm}$ satisfy the Parceval condition}
\begin{equation}
\sum_{l=0}^{\infty}\sum_{m=-l}^{m=l}|a_{lm}|^2=\int_{0}^{2\pi}\td \varphi
\int_{0}^{\pi} \td \vartheta \sin \vartheta |f(\vartheta,\varphi)|^2
\end{equation}
\par{The expansion \gref{multipole} in terms of the spherical harmonics is
  widely used in different branches of physics. It is called the {\it
  multipole expansion}, and $a_{lm}$ are called {\it multipole moments}.}
\par{Two expansions are widely used in physics world}
\begin{equation}
\label{cwave}
e^{i{\bf k}\cdot {\bf R}}=4\pi
\sum_{l=0}^{\infty}\sum_{m=-l}^{l}i^lj_l(kR)Y_{lm}(\hat{\bf R})Y_{lm}^{*}(\hat{\bf k})
\end{equation}
\begin{equation}
\label{coulomb}
\frac{1}{|{\bf r}_1-{\bf r}_2|}=\sum_{l=0}^{\infty}\sum_{m=-l}^{l}\frac{4 \pi}{2l+1}\frac{r_<^l}{r_>^{l+1}}Y_{lm}(\hat{{\bf r}_2})Y_{lm}^{*}(\hat{{\bf r}_1})
\end{equation}
where $r_<=\min (r_1,r_2)$ and $r_>=\max (r_1,r_2)$, $\hat{\bf x}$ stands for
the angular dependence of ${\bf x}$. If $r_1<r_2$ we get
\begin{equation}
\begin{split}
\frac{1}{|{\bf r}_1-{\bf r}_2|}=&\sum_{l=0}^{\infty}\sum_{m=-l}^{l}\frac{4 \pi}{2l+1}\frac{r_1^l}{r_2^{l+1}}Y_{lm}(\hat{{\bf r}_2})Y_{lm}^{*}(\hat{{\bf r}_1})\\=&\sum_{l=0}^{\infty}\sum_{m=-l}^{l}(-1)^m\frac{4 \pi}{2l+1}R_{l-m}({\bf r}_1)I_{lm}({\bf r}_2)
\end{split}
\end{equation}
%
\subsection[Expansion of Products...]{Expansion of Products of the Spherical Harmonics}
\par{A direct product of two spherical harmonics of the same argument may be
  expanded in series as (the so-called {\it Clebsch-Gordan series})}
\begin{equation}
\label{product2}
Y_{l_1m_1}(\vartheta,\varphi)Y_{l_2m_2}(\vartheta,\varphi)=\sum_{L=0}^{\infty}\sum_{M=-L}^{L}\sqrt{\frac{(2l_1+1)(2l_2+1)}{4\pi(2L+1)}}C_{l_10l_20}^{L0}C_{l_1m_1l_2m_2}^{LM}Y_{LM}(\vartheta,\varphi)
\end{equation}
\par{The inverse relation may be written as}
\begin{equation}
\begin{split}
C_{l_10l_20}^{L0}Y_{LM}(\vartheta,\varphi)=&\sqrt{\frac{(2l_1+1)(2l_2+1)}{4\pi(2L+1)}}\\&\times\sum_{m_1=-l_1}^{l_1}\sum_{m_2=-l_2}^{l_2}C_{l_1m_1l_2m_2}^{LM}Y_{l_1m_1}(\vartheta,\varphi)Y_{l_2m_2}(\vartheta,\varphi)
\end{split}
\end{equation}
\par{Product of three spherical harmonics can be decomposed as}
\begin{equation}
\begin{split}
Y_{l_1m_1}(\vartheta,\varphi)Y_{l_2m_2}(\vartheta,\varphi)Y_{l_3m_3}(\vartheta,\varphi)=&\sum_{L,L^{\prime},M,M^{\prime}}\sqrt{\frac{(2l_1+1)(2l_2+1)(2l_3+1)}{(4\pi)^2(2L+1)}}\\&\times C_{l_10l_20}^{L^{\prime}0}C_{L^{\prime}0l_30}^{L0}
C_{l_1m_1l_2m_2}^{L^{\prime}M^{\prime}}C_{L^{\prime}M^{\prime}l_2m_2}^{LM}Y_{LM}(\vartheta,\varphi)
\end{split}
\end{equation}
\subsection[Integrals ...]{Integrals over Total Solid Angle}
\begin{equation}
\int_{0}^{2\pi}\td \varphi\int_{0}^{\pi}\td \vartheta \sin \vartheta Y_{lm}(\vartheta,\varphi)=\sqrt{4\pi}\delta_{l0}\delta_{m0}
\end{equation}
\par{To get the result we multiply the integrand by $Y_{00}^{*}(\vartheta,\varphi)$ and
divide it by the same quantity. Taking into account that
$1/Y_{00}^{*}(\vartheta,\varphi)=\sqrt{4\pi}$ and using completeness relation
we get our result.}
\begin{equation}
\int_{0}^{2\pi}\td \varphi\int_{0}^{\pi}\td \vartheta \sin \vartheta Y_{l_1m_1}(\vartheta,\varphi)Y_{l_2m_2}(\vartheta,\varphi)=(-1)^{m_2}\delta_{l_1l_2}\delta_{m_1-m_2}
\end{equation}
\par{We replace $Y_{l_2m_2}(\vartheta,\varphi)$ with
$(1)^{m_2}Y_{l_2-m_2}(\vartheta,\varphi)$ and using completeness relation we
get the result.}
%
\begin{equation}
\label{gauntdef}
\begin{split}
\int_{0}^{2\pi}\td \varphi\int_{0}^{\pi}\td \vartheta \sin \vartheta
&Y_{l_1m_1}(\vartheta,\varphi)Y_{l_2m_2}(\vartheta,\varphi)Y_{l_3m_3}^{*}(\vartheta,\varphi)=\\=&\sqrt{\frac{(2l_1+1)(2l_2+1)}{4\pi(2l_3+1)}} C_{l_10l_20}^{l_30}C_{l_1m_1l_2m_2}^{l_3m_3}
:=G_{l_1m_1l_2m_2}^{l_3m_3}
\end{split}
\end{equation}
\par{Relation \gref{gauntdef} defines Gaunt coefficient $G_{l_1m_1l_2m_2}^{l_3m_3}$.}
\par{Replacing $Y_{l_1m_1}(\vartheta,\varphi)Y_{l_2m_2}(\vartheta,\varphi)$ with
the expansion \gref{product2} and using completeness relation we get our result.}
\begin{equation}
\label{yjm3}
\begin{split}
\int_{0}^{2\pi}\td \varphi&\int_{0}^{\pi}\td \vartheta \sin \vartheta
Y_{l_1m_1}(\vartheta,\varphi)Y_{l_2m_2}(\vartheta,\varphi)Y_{l_3m_3}(\vartheta,\varphi)=\\=&\sqrt{\frac{(2l_1+1)(2l_2+1)(2l_3+1)}{4\pi}}
\tjm{l_1}{0}{l_2}{0}{l_3}{0}\tjm{l_1}{m_1}{l_2}{m_2}{l_3}{m_3}\\=&(-1)^{m_3}G_{l_1m_1l_2m_2}^{l_3-m_3}
\end{split}
\end{equation}
%
where $\binom{l_1\quad l_2\quad l_3}{m_1\quad m_2\quad m_3}$ is Wigner
$3jm$-symbol.To prove relation relation \gref{yjm3} we just replace
$Y_{l_3m_3}(\vartheta,\varphi)$ with
$(-1)^{m_3}Y_{l_3-m_3}^{*}(\vartheta,\varphi)$. Then we apply
relation \gref{gauntdef} and get our result.
%
\par{ The connection with Clebsch-Gordon coefficients is}
\begin{equation}
\tjm{j_1}{m_1}{j_2}{m_2}{j_3}{m_3}=(-1)^{j_3+m_3+2j_1}\frac{1}{\sqrt{2j_3+1}}C_{j_1-m_1j_2-m_2}^{j_3m_3}
\end{equation}
\par{The inverse relation is}
\begin{equation}
C_{j_1m_1j_2m_2}^{j_3m_3}=(-1)^{j_1-j_2+m_3}\sqrt{2j_3+1}\tjm{j_1}{m_1}{j_2}{m_2}{j_3}{-m_3}
\end{equation}
\par{The $3jm$-symbol represents the probability amplitude that three angular
momenta ${\bf j}_1,{\bf j}_3,{\bf j}_3$ with projections $m_1,m_2,m_3$ are
coupled to yields zero angular momentum}
\begin{equation}
\tjm{j_1}{m_1}{j_2}{m_2}{j_3}{m_3}=(-1)^{j_1-j_2+j_3}\sum_{jm}C_{j_1m_1j_2m_2}^{jm}C_{jmj_3m_3}^{00}
\end{equation}
%
\subsection{Solid Spherical Harmonics}
\par{Solid harmonics regular and irregular are solutions of laplace equation,
  hence their study is of a great interest. The addition theorems are}
\begin{equation}
R_{lm}(\bm{a}+\bm{b})=\sum_{l_1=0}^{l}\sum_{m=-l_1}^{l_1}\frac{4\pi(2l+1)!!}{(2l_1+1)!!(2l_2+1)!!}\mc{G}_{l_1m_1l_2m_2}^{lm}R_{l_1m_1}(\bm{a})R_{l_2m_2}(\bm{b})
\end{equation}
with $l=l_1+l_2$ and $m=m_1+m_2$
\begin{equation}
I_{lm}({\bf a}+\bm{b})=\sum_{l_1,m_1}\frac{4\pi(2l_1-1)!!}{(2l-1)!!(2l_2+1)!!}(-1)^{l_2}R_{l_2m_2}^{*}({\bf a})I_{l_1m_1}({\bm{b}})G_{lml_2m_2}^{l_1m_1}
\end{equation}
with $l_1=l+l_2$ and $m_1=m+m_2$
\par{The proofs of these theorems follow the line of \citep{Chakrabarti95},\citep{Deb83}.}
\par{\textbf{Regular solid harmonics. }
We start with expansion \gref{cwave}. We multiply it by $Y_{l_1m_1}(\hat{\bm{k}})$
and integrate over $\bm{k}$ angles. Using the orthonormality relation we get}
\begin{equation}
\begin{split}
\int \td {\hat{\bm{k}}}e^{i{\bf k}\cdot {\bf R}}Y_{l_1m_1}(\hat{\bm{k}})&=4\pi
\sum_{l=0}^{\infty}\sum_{m=-l}^{l}i^lj_l(kR)Y_{lm}(\hat{\bf R})\int \td
{\hat{\bm{k}}} Y_{lm}^{*}(\hat{\bf k})Y_{l_1m_1}(\hat{\bm{k}})\\
Q&=4\pi
\sum_{l=0}^{\infty}\sum_{m=-l}^{l}i^lj_l(kR)Y_{lm}(\hat{\bf R})\delta_{ll_1}\delta_{mm_1}
\end{split}
\end{equation}
\begin{equation}
\label{p1}
Q=4\pi i^lj_l(kR)Y_{lm}(\hat{\bf R})
\end{equation} 
where $Q$ stands for rihgt hand side and $\td {\hat{\bm{k}}}=\sin \vartheta
\td \vartheta \td \varphi$. Next, let us consider
$\bm{R}=\bm{a}+\bm{b}$ then we have $e^{i{\bf k}\cdot {\bf R}}=e^{i{\bf
    k}\cdot {\bf a}}e^{i{\bf k}\cdot {\bf b}}$ and inserting one expansion
\gref{cwave} for each exponential we get
\begin{equation}
\begin{split}
e^{i{\bf k}\cdot {\bf R}}=(4\pi)^2
\sum_{l_1=0}^{\infty}\sum_{m_1=-l_1}^{l_1}\sum_{l_2=0}^{\infty}\sum_{m_2=-l_2}^{l_2}i^{l_1+l_2}j_{l_1}(ka)j_{l_2}(kb)&Y_{l_1m_1}(\hat{\bf a})Y_{l_1m_1}^{*}(\hat{\bf k})\\&\times Y_{l_2m_2}(\hat{\bf b})Y_{l_2m_2}^{*}(\hat{\bf k})
\end{split}
\end{equation}
then multiplying it by $Y_{lm}(\hat{\bm{k}})$
and integrating over $\bm{k}$ angles we get
\begin{equation}
\label{p2}
Q=(4\pi)^2
\sum_{l_1=0}^{\infty}\sum_{m_1=-l_1}^{l_1}\sum_{l_2=0}^{\infty}i^{l_1+l_2}j_{l_1}(ka)j_{l_2}(kb)Y_{l_1m_1}(\hat{\bf a})Y_{l_2m_2}(\hat{\bf b})G_{l_1m_1l_2m_2}^{lm}
\end{equation}
with $G_{l_1m_1l_2m_2}^{lm}$ the Gaunt coefficient. Summation over $m_2$
disappers due to selection rule of Gaunt coefficients, $m_2=m-m_1$. $l$
obbeys the triangle inequality $|l_1-l_2|\le l\le |l_1+l_2|$. 
\par{Combining \gref{p1} with \gref{p2} we get}
\begin{equation}
i^lj_l(kR)Y_{lm}(\hat{\bf R})=4\pi\sum_{l_1=0}^{\infty}\sum_{m_1=-l_1}^{l_1}\sum_{l_2=0}^{\infty}i^{l_1+l_2}j_{l_1}(ka)j_{l_2}(kb)Y_{l_1m_1}(\hat{\bf a})Y_{l_2m_2}(\hat{\bf b})G_{l_1m_1l_2m_2}^{lm}
\end{equation}
using the behaviour of $j_l(kR)$ when $k\rightarrow 0$,
$j_l(kR)\xrightarrow{k\rightarrow 0}\frac{(kR)^l}{(2l+1)!!}$ we get
\begin{equation}
\frac{i^lR^{l}}{(2l+1)!!}Y_{lm}(\hat{\bf R})=4\pi\sum_{l_1=0}^{\infty}\sum_{m_1=-l_1}^{l_1}\sum_{l_2=0}^{\infty}\frac{i^{l_1+l_2}k^{l_1+l_2-l}a^{l_1}b^{l_2}}{(2l_1+1)!!(2l_2+1)!!}Y_{l_1m_1}(\hat{\bf a})Y_{l_2m_2}(\hat{\bf b})G_{l_1m_1l_2m_2}^{lm}
\end{equation}
and because in the limit  $k\rightarrow 0$ left hand side should be always
finite we get $l=l_1+l_2$ and relation becomes
\begin{equation*}
R_{lm}(\bm{R})=R_{lm}(\bm{a}+\bm{b})=4\pi\sum_{l_1=0}^{l}\sum_{m_1=-l_1}^{l_1}\frac{(2l+1)!!}{(2l_1+1)!!(2l_2+1)!!}R_{l_1m_1}({\bf a})R_{l_2m_2}({\bf b})G_{l_1m_1l_2m_2}^{lm}    
\end{equation*}
with $l_2=l-l_1$ and $m_2=m-m_1$ this is the addition theorem for regular
spherical harmonics.
\par{We also could find useful the following properties}
\begin{equation}
\begin{split}
R_{lm}(-\bm{r})&=r^lY_{lm}(\pi
-\vartheta,\pi+\varphi)=(-1)^{l}r^lY_{lm}(\vartheta,\varphi)=(-1)^lR_{lm}(\bm{r})\\
R_{lm}^{*}(r)&=r^lY_{lm}^{*}(\vartheta,\varphi)=(-1)^mr^lY_{l-m}(\vartheta,\varphi)=(-1)^mR_{l-m}(r)
\end{split}
\end{equation}
\par{\textbf{Irregular solid harmonics.} To prove the addition theorem for irregular solid harmonics we use
  expansion \gref{coulomb}(with the assumption $r_1<r_2$). We multiply it by
  $Y_{l_1m_1}(\hat{\bm{r}_1})$ and integrate over $\bm{r}_1$ angles. Using the
  orthonormality relation for spherical harmonics we get}
\begin{equation}
\begin{split}
\int \td \hat{\bm {r}_1} \frac{Y_{l_1m_1}(\hat{{\bf r}_1})}{|{\bf r}_1-{\bf
  r}_2|}&=\sum_{l=0}^{\infty}\sum_{m=-l}^{l}\frac{4
  \pi}{2l+1}\frac{r_1^l}{r_2^{l+1}}Y_{lm}(\hat{{\bf r}_2})\int \td \hat{\bm
  {r}_1} Y_{l_1m_1}(\hat{{\bf r}_1})Y_{lm}^{*}(\hat{{\bf r}_1})\\
Q&=\sum_{l=0}^{\infty}\sum_{m=-l}^{l}\frac{4
  \pi}{2l+1}\frac{r_1^l}{r_2^{l+1}}Y_{lm}(\hat{{\bf r}_2})\delta_{ll_1}\delta_{mm_1}
\end{split}
\end{equation}
\begin{equation}
\label{q1}
Q=\frac{4
  \pi}{2l+1}\frac{r_1^l}{r_2^{l+1}}Y_{lm}(\hat{{\bf r}_2})=\frac{4
  \pi}{2l+1}r_1^lI_{lm}({\bf r}_2)
\end{equation}
where $Q$ stands for the right hand side of the first line.
\par{Next let us consider $\bm{r}_2=\bm{a}+\bm{b}$, we assume that
  $|r_1-a|<b$}. In this conditions we get the following expansion
\begin{equation}
\frac{1}{|{\bf r}_1-{\bf r}_2|}=\frac{1}{|(\bm{r}_1-\bm{a})-\bm{b}|}=\sum_{l_1,m_1}\frac{4\pi}{2l_1+1}R_{l_1m_1}^{*}(\bm{r}_1-\bm{a})I_{l_1m_1}({\bm{b}})
\end{equation}
we used the definition of regular and irregular solid harmonics to write the
above relation in a condensed way. Next we apply addition theorem
for regular solid harmonics and we get
\begin{equation}
\begin{split}
\frac{1}{|(\bm{r}_1-\bm{a})-\bm{b}|}=\sum_{l_1,m_1}I_{l_1m_1}({\bm{b}})\sum_{l_2=0}^{l_1}\sum_{m_2=-l_2}^{l_2}&\frac{(4\pi)^2(2l_1-1)!!r_1^{l_2}}{(2l_2+1)!!(2l_3+1)!!}Y_{l_2m_2}^{*}(\hat{{\bf r}_1})\\&\times (-1)^{l_3}R_{l_3m_3}^{*}({\bf a})G_{l_2m_2l_3m_3}^{l_1m_1}
\end{split}
\end{equation}
\par{Multiplying it by $Y_{lm}(\hat{{\bf r}_1})$, integrating over ${\bf r}_1$
angles and finally using the orthonormality relation for spherical harmonics
we get}
\begin{equation}
\label{q2}
Q=\sum_{l_1,m_1}I_{l_1m_1}({\bm{b}})\frac{(4\pi)^2(2l_1-1)!!}{(2l+1)!!(2l_3+1)!!}(r_1)^{l}(-1)^{l_3}R_{l_3m_3}^{*}({\bf a})G_{lml_3m_3}^{l_1m_1}
\end{equation}
with $l_1=l+l_3$ and $m_1=m+m_3$
\par{Combining \gref{q1} with \gref{q2} we get}
\begin{equation}
\frac{1}{2l+1}r_1^lI_{lm}({\bf r}_2)=\sum_{l_1,m_1}I_{l_1m_1}({\bm{b}})\frac{4\pi(2l_1-1)!!}{(2l+1)!!(2l_3+1)!!}r_1^{l}(-1)^{l_3}R_{l_3m_3}^{*}({\bf a})G_{lml_3m_3}^{l_1m_1}
\end{equation}
and equating the coefficients of ${r}_1^{l}$
\begin{equation*}
I_{lm}({\bf r}_2)=I_{lm}({\bf a}+\bm{b})=\sum_{l_1,m_1}\frac{4\pi(2l_1-1)!!}{(2l-1)!!(2l_3+1)!!}(-1)^{l_3}R_{l_3m_3}^{*}({\bf a})I_{l_1m_1}({\bm{b}})G_{lml_3m_3}^{l_1m_1}
\end{equation*}
which is the addition theorem for irregular solid harmonics. The theorem was
obtained under some restrictions, but is very simple to prove it if we state the
opposite of this conditions following the same line.
\par{Following properties are useful}
\begin{equation}
\begin{split}
I_{lm}(-\bm{r})&=(-1)^lI_{lm}(\bm{r})\\
I_{lm}^{*}(\bm{r})&=(-1)^mI_{l-m}(\bm{r})
\end{split}
\end{equation} 
\section[Real Spherical ...]{Real Spherical Harmonics}
\subsection{Definition}
\par{We can define real spherical harmonics as}
\begin{equation}
\begin{split}
X_{l0}&:=Y_{l0}\\ m>0:\quad
X_{lm}&:=\sqrt{2}(-1)^m\text{Re}Y_{lm}\\m<0:\quad X_{lm}&:=\sqrt{2}(-1)^m\text{Im}Y_{l-m}
\end{split}\end{equation}
or in a condensed form
\begin{equation}
\label{Rspherdef}
X_{l\mu}(\vartheta,\varphi)=\sum_{m=-l}^{l}U_{lm}^{\mu}Y_{lm}(\vartheta,\varphi)
\end{equation}
where $\mu=\overline{-l...l}$, $U_{lm}^{\mu}$ represents a $(2l+1)\times (2l+1)$
unitary matrix for a fixed $l$ and has the
elements
\begin{equation}
U_{lm}^{\mu}=\delta_{m0}\delta_{\mu 0}+\frac{1}{\sqrt{2}}\left((-1)^m
  \Theta(\mu)\delta_{m\mu}+i\Theta(-\mu)\delta_{m\mu}-i(-1)^m \Theta(-\mu)\delta_{m-\mu}+\Theta(\mu)\delta_{m-\mu}\right)
\end{equation}
\par{$\Theta(\mu)$} is given by
\begin{equation}
\Theta(\mu)=\left\{\begin{array}{ll}
1&\textrm{if \(\mu>0\)}\\
0&\textrm{if \(\mu \le 0\)}
\end{array}\right.
\end{equation}
%
\par{We have}
\begin{equation}
\label{unit}
\sum_{m}[U_{lm}^{\mu}]^{*}U_{lm}^{\mu^{\prime}}=\delta_{\mu \mu^{\prime}}\qquad\sum_{\mu}[U_{lm}^{\mu}]^{*}U_{lm^{\prime}}^{\mu}=\delta_{m m^{\prime}}
\end{equation}
\par{The inverse relation of \gref{Rspherdef} is}
\begin{equation}
Y_{lm}(\vartheta,\varphi)=\sum_{\mu}[U_{lm}^{\mu}]^{*}X_{l\mu}(\vartheta,\varphi)
\end{equation}
\subsection{Useful Properties}
\par{Starting with obvious equality $[X_{l\mu}]^{*}=X_{l\mu}$ and expanding each
side in terms of \gref{Rspherdef} we get}
\begin{equation*}
\sum_{m=-l}^{l}U_{lm}^{\mu}Y_{lm}(\vartheta,\varphi)=\sum_{m=-l}^{l}[U_{lm}^{\mu}]^{*}Y_{lm}^{*}(\vartheta,\varphi)
\end{equation*}
\begin{equation}
\sum_{m=-l}^{l}\left\{(-1)^m[U_{l-m}^{\mu}]^{*}-U_{lm}^{\mu}\right\}Y_{lm}(\vartheta,\varphi)=0
\end{equation}
\par{Since the spherical harmonics are linearly independent for a fixed $l$ the
content of the curly brackets should vanish, so we get}
\begin{equation}
\label{ulm}
U_{lm}^{\mu}=(-1)^m[U_{l-m}^{\mu}]^{*}\qquad[U_{lm}^{\mu}]^{*}=(-1)^mU_{l-m}^{\mu}
\end{equation}
\par{A very useful relation is}
\begin{equation}
\label{sumxx}
\sum_{\mu}X_{l\mu}(\vartheta_1,\varphi_1)X_{l\mu}(\vartheta_2,\varphi_2)=\sum_{m}Y_{lm}(\vartheta_1,\varphi_1)Y_{lm}^{*}(\vartheta_2,\varphi_2)
\end{equation}
\par{The proof is straightforward. In left hand side we insert for each real
spherical harmonic an expansion \gref{Rspherdef} and we get}
\begin{equation}
\begin{split}
\sum_{\mu}X_{l\mu}(\vartheta_1,\varphi_1)&X_{l\mu}(\vartheta_2,\varphi_2)=\sum_{\mu}\sum_{m}U_{lm}^{\mu}Y_{lm}(\vartheta_1,\varphi_1)\sum_{m^{\prime}}U_{lm^{\prime}}^{\mu}Y_{lm^{\prime}}(\vartheta_2,\varphi_2)\\
&=\sum_{m}\sum_{m^{\prime}}(-1)^{m\prime}\sum_{\mu}[U_{l-m^{\prime}}^{\mu}]^{*}U_{lm}^{\mu}Y_{lm}(\vartheta_1,\varphi_1)Y_{lm^{\prime}}(\vartheta_2,\varphi_2)\\
&=\sum_{m}\sum_{m^{\prime}}(-1)^{m\prime}\delta_{m-m^{\prime}}Y_{lm}(\vartheta_1,\varphi_1)Y_{lm^{\prime}}(\vartheta_2,\varphi_2)\\&=\sum_{m}Y_{lm}(\vartheta_1,\varphi_1)Y_{lm}^{*}(\vartheta_2,\varphi_2)
\end{split}
\end{equation}
\par{To get second line we use \gref{ulm} and then in second we make use of
\gref{unit} and then using phase convention for spherical harmonics we get
right hand side of \gref{sumxx}.}
\par{The ortho-normalization relation for real spherical harmonics is easy to
  get.}
\begin{equation}
\begin{split}
\int_{0}^{2\pi}\td \varphi&  \int_{0}^{\pi}\td \vartheta \sin \vartheta
X_{l\mu}(\vartheta,\varphi)X_{l_1\mu_1}(\vartheta,\varphi)\\&=\sum_{m,m_1}U_{lm}^{\mu}U_{l_1m_1}^{\mu_1}\int_{0}^{2\pi}\td
\varphi \int_{0}^{\pi}\td \vartheta \sin \vartheta
Y_{lm}(\vartheta,\varphi)Y_{l_1m_1}(\vartheta,\varphi)\\
&=\sum_{m,m_1}U_{lm}^{\mu}U_{l_1m_1}^{\mu_1}(-1)^m\int_{0}^{2\pi}\td
\varphi \int_{0}^{\pi}\td \vartheta \sin \vartheta Y_{l-m}^{*}(\vartheta,\varphi)Y_{l_1m_1}(\vartheta,\varphi)\\&=\delta_{l,l_1}\sum_{m,m_1}U_{lm}^{\mu}U_{l_1m_1}^{\mu_1}(-1)^m\delta_{-m,m_1}=\delta_{l,l_1}\sum_{m}U_{lm}^{\mu}U_{l-m}^{\mu_1}(-1)^m\\&=\delta_{l,l_1}\sum_{m}U_{lm}^{\mu}[U_{lm}^{\mu_1}]^{*}=\delta_{l,l_1}\delta_{\mu,\mu_1}
\end{split}
\end{equation}
\par{In first line we replace each real spherical harmonics by an expansion
\gref{Rspherdef}. In second line we create the ortho-normalization relation for
complex spherical harmonics using phase convention relation. In forth line we
made use of \gref{unit} and we get our final result. The real spherical
harmonics keep the same form as complex one for the ortho-normalization
relation. Next let us check the completeness relation}
\begin{equation}
\begin{split}
\sum_{l,\mu}X_{l\mu}(\vartheta_1,\varphi_1)&X_{l\mu}(\vartheta_2,\varphi_2)=\sum_{l,\mu,m_1,m_2}U_{lm_1}^{\mu}U_{lm_2}^{\mu}Y_{lm_1}(\vartheta_1,\varphi_1)Y_{lm_2}(\vartheta_2,\varphi_2)\\&=\sum_{l,m_1,m_2}\sum_{\mu}[U_{l-m_1}^{\mu}]^{*}U_{lm_2}^{\mu}Y_{l-m_1}^{*}(\vartheta_1,\varphi_1)Y_{lm_2}(\vartheta_2,\varphi_2)\\&=\sum_{l,m_1,m_2}\delta_{-m_1,m_2}Y_{l-m_1}^{*}(\vartheta_1,\varphi_1)Y_{lm_2}(\vartheta_2,\varphi_2)\\&=\sum_{l,m}Y_{lm}^{*}(\vartheta_1,\varphi_1)Y_{lm}(\vartheta_2,\varphi_2)=\delta(\varphi_1-\varphi_2)\delta(\cos \vartheta_1-\cos \vartheta_2)
\end{split}
\end{equation}
\par{The line of proof is identical with that one followed for ortho-normalization relation.}
\subsection[Expansions ...]{Expansions in terms of real spherical harmonics}
\begin{equation}
\frac{1}{|{\bf r}_1-{\bf
    r}_2|}=\sum_{l=0}^{\infty}\sum_{m=-l}^{l}\frac{4\pi}{2l+1}\frac{r_<^l}{r_>^{l+1}}Y_{lm}(\hat{{\bf r}_1})Y_{lm}^{*}(\hat{{\bf r}_2})
\end{equation}
with $r_<=\min (r_1,r_2)$ and $r_>=\max (r_1,r_2)$.
\begin{equation}
\begin{split}
\frac{1}{|{\bf r}_1-{\bf
    r}_2|}&=\sum_{l=0}^{\infty}\sum_{m=-l}^{l}\frac{4\pi}{2l+1}\frac{r_<^l}{r_>^{l+1}}Y_{lm}(\hat{{\bf
    r}_1})Y_{lm}^{*}(\hat{{\bf r}_2})\\
&=\sum_{l=0}^{\infty}\sum_{m=-l}^{l}\frac{4\pi}{2l+1}\frac{r_<^l}{r_>^{l+1}}\sum_{\mu_1}\left[
    U_{lm}^{\mu_1}\right]^{*}X_{l\mu_1}(\hat{{\bf r}_1})\sum_{\mu_2}
    U_{lm}^{\mu_2}X_{l\mu_2}(\hat{{\bf r}_2})\\
&=\sum_{l=0}^{\infty}\frac{4\pi}{2l+1}\frac{r_<^l}{r_>^{l+1}}\sum_{\mu_1,\mu_2}\underbrace{\sum_{m=-l}^{l}\left[
    U_{lm}^{\mu_1}\right]^{*}U_{lm}^{\mu_2}}_{\delta_{\mu_1\mu_2}}X_{l\mu_1}(\hat{{\bf r}_1})X_{l\mu_2}(\hat{{\bf r}_2})
\end{split}
\end{equation}
\par{In the second line we replaced the complex spherical harmonics with expansions
in terms of the real counterparts. In the third line we use relation
\gref{unit} and we get the expansion}
\begin{equation}
\label{rcoulomb}
\frac{1}{|{\bf r}_1-{\bf
    r}_2|}=\sum_{l=0}^{\infty}\sum_{m=-l}^{l}\frac{4\pi}{2l+1}\frac{r_<^l}{r_>^{l+1}}X_{lm}(\hat{{\bf
    r}_1})X_{lm}(\hat{{\bf r}_2})
\end{equation}
for $r_1<r_2$ we have
\begin{equation}
 \frac{1}{|{\bf r}_1-{\bf
    r}_2|}=\sum_{l=0}^{\infty}\sum_{m=-l}^{l}\frac{4\pi}{2l+1}\mc{R}_{lm}({\bf
    r}_1)\mc{I}_{lm}({\bf r}_2)
\end{equation}
and for $r_1>r_2$
\begin{equation}
 \frac{1}{|{\bf r}_1-{\bf
    r}_2|}=\sum_{l=0}^{\infty}\sum_{m=-l}^{l}\frac{4\pi}{2l+1}\mc{I}_{lm}({\bf
    r}_1)\mc{R}_{lm}({\bf r}_2)
\end{equation}
where we defined $\mc{R}_{lm}$ and $\mc{I}_{lm}$ as
\begin{equation}
\mc{R}_{lm}({\bf r})=r^lX_{lm}(\hat{{\bf r}})\qquad \mc{I}_{lm}({\bf r})=r^{-l-1}X_{lm}(\hat{{\bf r}})
\end{equation}
\begin{equation}
e^{i\bm{r}_1\cdot\bm{r}_2}=4\pi\sum_{l=0}^{\infty}\sum_{m=-l}^{l}i^lj_l(r_1r_2)Y_{lm}(\hat{\bm{r}}_2)Y_{lm}^{*}(\hat{\bm{r}}_1)
\end{equation}
and following the same line as above we find out the following expansion
\begin{equation}
\label{rwave}
e^{i\bm{r}_1\cdot\bm{r}_2}=4\pi\sum_{l=0}^{\infty}\sum_{m=-l}^{l}i^lj_l(r_1r_2)X_{lm}(\hat{\bm{r}}_2)X_{lm}(\hat{\bm{r}}_1)
\end{equation}
\subsection[Addition Theorem ...]{Addition Theorem for Real Solid Harmonics}
\par{Relation \gref{rwave} could be used to get an addition theorem for
  $\mc{R}_{lm}$ as suggested in \citep{Chakrabarti95},\citep{Deb83}}
\par{Multiplying relation \gref{rwave} by $X_{l_1m_1}(\hat{\bm{r}}_1)$ and
  integrating over $\td \hat{\bm{r}}_1$ we get}
\begin{equation}
\label{part1}
\begin{split}
Q&=\int \td \hat{\bm{r}}_1
e^{i\bm{r}_1\cdot\bm{r}_2}X_{l_1m_1}(\hat{\bm{r}}_1)=4\pi\sum_{l=0}^{\infty}\sum_{m=-l}^{l}i^lj_l(r_1r_2)X_{lm}(\hat{\bm{r}}_2)\int \td \hat{\bm{r}}_1
X_{lm}(\hat{\bm{r}}_1)X_{l_1m_1}(\hat{\bm{r}}_1)\\
Q&=4\pi\sum_{l=0}^{\infty}\sum_{m=-l}^{l}i^lj_l(r_1r_2)X_{lm}(\hat{\bm{r}}_2)\delta_{ll_1}\delta_{mm_1}\\
Q&=4\pi i^lj_l(r_1r_2)X_{lm}(\hat{\bm{r}}_2)
\end{split}
\end{equation}
\par{Considering $\bm{r}=\bm{a}+\bm{b}$ we get
$e^{i\bm{r}_1\cdot\bm{r}_2}=e^{i\bm{r}_1\cdot\bm{a}}e^{i\bm{r}_1\cdot\bm{b}}$
and inserting an expansion \gref{rwave} for each exponential, and then,
multiplying relation by $X_{lm}(\hat{\bm{r}}_1)$ and integrating over $\td \hat{\bm{r}}_1$ we get}
\begin{equation}
\begin{split}
e^{i\bm{r}_1\cdot(\bm{a}+\bm{b})}=(4\pi)^2\sum_{l_1,m_1}i^{l_1}j_{l_1}(r_1a)X_{l_1m_1}(\hat{\bm{a}})X_{l_1m_1}(\hat{\bm{r}}_1)\sum_{l_2,m_2}i^{l_2}j_{l_2}(r_1b)X_{l_2m_2}(\hat{\bm{b}})X_{l_2m_2}(\hat{\bm{r}}_1)\\
Q=\sum_{\begin{subarray}{l}l_1,m_1\\
l_2,m_2
\end{subarray}
}i^{l_1+l_2}j_{l_1}(r_1a)j_{l_2}(r_1b)X_{l_1m_1}(\hat{\bm{a}})X_{l_2m_2}(\hat{\bm{b}})\int
\td \hat{\bm{r}}_1 X_{l_1m_1}(\hat{\bm{r}}_1)X_{l_2m_2}(\hat{\bm{r}}_1)X_{lm}(\hat{\bm{r}}_1)
\end{split}
\end{equation}
\par{The integral is nothing else than Gaunt coefficient for real spherical
  harmonics $\mc{G}_{l_1m_1l_2m_2}^{lm}=\int \td \hat{\bm{r}}_1
  X_{l_1m_1}(\hat{\bm{r}}_1)X_{l_2m_2}(\hat{\bm{r}}_1)X_{lm}(\hat{\bm{r}}_1)$.
  Using }
\begin{equation}
j_l(r_1r_2)\xrightarrow{r_1\rightarrow 0}\frac{(r_1r_2)^l}{(2l+1)!!}
\end{equation}
and replacing $Q$ with relation \gref {part1} we get
\begin{equation}
i^l(r_1r_2)^l=\sum_{\begin{subarray}{l}l_1,m_1\\
l_2,m_2
\end{subarray}}i^{l_1+l_2}\frac{4\pi(2l+1)!!}{(2l_1+1)!!(2l_2+1)!!}
r_1^{l_1+l_2}a^{l_1}X_{l_1m_1}(\hat{\bm{a}})b^{l_2}X_{l_2m_2}(\hat{\bm{b}})\mc{G}_{l_1m_1l_2m_2}^{lm}
\end{equation}
\par{Using the selection rules for real Gaunt coefficients and the fact that the
expansion should be finite, $m_2=m-m_1$ and $l_2=l-l_1$, we extract from above
expansion the addition rule for $\mc{R}_{lm}$}
\begin{equation}
\label{radd}
\mc{R}_{lm}(\bm{a}+\bm{b})=\sum_{l_1=0}^{l}\sum_{m_1=-l_1}^{l_1}\mc{G}_{l_1m_1l_2m_2}^{lm}\frac{4\pi(2l+1)!!}{(2l_1+1)!!(2l_2+1)!!}\mc{R}_{l_1m_1}(\bm{a})\mc{R}_{l_2m_2}(\bm{b})
\end{equation}
\par{ A similar theorem could be obtained for real irregular solid
  harmonics. We start with expansion \gref{rcoulomb}, for $r_1<r_2$ and
  multiplying it by $X_{l_1m_1}(\hat{\bm{r}_1})$, integrating over $\bm{r}_1$
  angles and using orthonormality relation for real spherical harmonics we
  get} 
\begin{equation}
\label{q4}
\begin{split}
\int \td \hat{\bm{r}}_1\frac{X_{l_1m_1}(\hat{\bm{r}_1})}{|{\bf r}_1-{\bf
    r}_2|}&=\sum_{l=0}^{\infty}\sum_{m=-l}^{l}\frac{4\pi}{2l+1}\frac{r_1^l}{r_2^{l+1}}X_{lm}(\hat{{\bf r}_2}) \int \td \hat{\bm{r}}_1 X_{lm}(\hat{{\bf r}_1})X_{l_1m_1}(\hat{{\bf r}_1})\\
Q&=\frac{4\pi}{2l+1}r_1^l\mc{I}_{lm}({\bf r}_2)
\end{split}
\end{equation}
with $Q$ standing for right hand side of first line and we used the definition
of $\mc{I}$ to write the last line.
\par{Let us consider, $\bm{r}_2=\bm{a}+\bm{b}$, assume that
  $|\bm{r}_1-\bm{a}|<b$ and we get}
\begin{equation}
\frac{1}{|({\bf r}_1-{\bf
    a})-\bm{b}|}=\sum_{\lambda,\mu}\frac{4\pi}{2\lambda+1}\mc{I}_{\lambda \mu}(\bm{b})\mc{R}_{\lambda \mu}({\bf r}_1-{\bf a})
\end{equation}
and using \gref{radd} we get
\begin{equation}
\begin{split}
\frac{1}{|({\bf r}_1-{\bf
    a})-\bm{b}|}=\sum_{\lambda,\mu}\mc{I}_{\lambda \mu}(\bm{b})\sum_{l_1=0}^{\lambda}\sum_{m_1=-l_1}^{l_1}&\mc{G}_{l_1m_1l_2m_2}^{\lambda \mu}\frac{(4\pi)^2(2\lambda-1)!!}{(2l_1+1)!!(2l_2+1)!!}\mc{R}_{l_1m_1}(\bm{r}_1)\\&\times(-1)^{l_2}\mc{R}_{l_2m_2}(\bm{a})
\end{split}
\end{equation}
with $\lambda=l_1+l_2$ and $\mu=m_1+m_2$. Next multiplying it by $X_{lm}(\hat{\bm{r}_1})$, integrating over
$\bm{r}_1$ angles and using orthonormality relation for real spherical
harmonics we get 
\begin{equation}
\label{q3}
Q=\sum_{\lambda,\mu}\mc{I}_{\lambda \mu}(\bm{b})\mc{G}_{lml_2m_2}^{\lambda \mu}\frac{(4\pi)^2(2\lambda-1)!!}{(2l+1)!!(2l_2+1)!!}r_1^l(-1)^{l_2}\mc{R}_{l_2m_2}(\bm{a})
\end{equation}
\par{Combining \gref{q3} with \gref{q4} and equating the coefficients with the same
power of $r_1$ we have}
\begin{equation}
\label{iadd}
\mc{I}_{lm}({\bf a}+\bm{b})=\sum_{\lambda,\mu}\mc{I}_{\lambda \mu}(\bm{b})\mc{G}_{lml_2m_2}^{\lambda \mu}\frac{4\pi(2\lambda-1)!!}{(2l-1)!!(2l_2+1)!!}(-1)^{l_2}\mc{R}_{l_2m_2}(\bm{a})
\end{equation}
with $l_2=\lambda-l$ and $m_2=\mu-m$. 
\par{The relation \gref{iadd} was obtained
under some restrictions. It is very simple to prove that is true if we state
the opposite following the same line of proof. It represents the addition
theorem for real irregular harmonics.}